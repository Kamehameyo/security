\documentclass{article}
\usepackage[french]{babel}
\usepackage{fullpage}
\usepackage[utf8]{inputenc}

\usepackage{listings}

\title{OPENSSL}

\begin{document}
\maketitle
\section{Génération de nombres aléatoires}

Pour générer des valeurs aléatoires on dispose de la commande \texttt{rand} dans \texttt{openssl}: 
\begin{lstlisting}[language=bash]
$ openssl rand -out dest num

\end{lstlisting}

On a les options suivantes:
\begin{itemize}
\item \texttt{num} est le nombre d’octets aléatoires produits.
\item \texttt{-out} dest est le nom du fichier de destination.
\end{itemize}

On dispose aussi d’options pour formater la sortie en hexadécimal ou en base64.

\subsection*{Questions}
\begin{itemize}

\item Utiliser \texttt{openssl} pour produire une clef et un vecteur d’initialisation de longueur respective
128 bits sauvegarder dans \texttt{key.rnd} et \texttt{iv.rnd}.
\item En utilisant, la documentation de \texttt{openssl} expliquer le processus qui a été mis en œuvre pour générer ces nombres. Est-ce satisfaisant ?



\item Sous Linux, on dispose aussi \texttt{/dev/random} et \texttt{/dev/random}

\begin{lstlisting}[language=bash]
$ dd if=/dev/urandom of=data.rnd bs=16 count=1
\end{lstlisting}

Quelle est la différence entre \texttt{/dev/random} et \texttt{/dev/urandom} ?


\end{itemize}


\section{Chiffrement symétrique}

\subsection{Avec clef et IV}

On peut chiffrer ou déchiffrer des fichiers avec \texttt{openssl} en utilisant les commandes \texttt{-enc -c} et
\texttt{enc -d}:
\begin{lstlisting}[language=bash]
$ openssl enc -e -ciphername -in clear -out cipher -K mykey -iv my iv
\end{lstlisting}
avec les options suivantes:
\begin{itemize}

\item \texttt{-ciphername},
\item \texttt{-in} indique le nom du fichier à chiffrer,
\item \texttt{-out} indique le nom du fichier chiffré,
\item \texttt{-K} la clef en héxadecimal,
\item \texttt{-iv} le vecteur d’initialisation en héxadecimal.
\end{itemize}


\subsection*{Questions}
\begin{itemize}

\item Chiffrer le fichier \texttt{test1} avec l’AES en mode CTR (128 bits) avec la clef et le vecteur
d’initialisation des questions précédentes.
\item Même chose avec l’AES en mode CBC (128 bits). Comparer la taille du fichier \texttt{test1} avec
les deux fichiers chiffrés obtenus (\texttt{ls} ou \texttt{wc}). Est ce que cela correspond à vos attentes ?
\item Comparer la performance (vitesse) des différents modes de chiffrement que nous avons vu en
cours pour l’AES (128 bits) en utilisant l’option:
\begin{lstlisting}[language=bash]
$ openssl speed ciphername
\end{lstlisting}

Est ce que cela correspond à vos attentes en terme de performance ?


\item Le fichier \texttt{test2.enc} a été chiffré. En analysant le fichier avec \texttt{hexdump} quel mode opératoire
a-t-il été utilisé ?
\item Les fichiers \texttt{test3.enc} et \texttt{test4.enc} ont été chiffrée avec l’AES et la même clef. Quelle erreur
a été commise lors du chiffrement ?
\end{itemize}

\subsection{Avec mot de passe}
Il est relativement difficile pour un être humain de retenir la clef et le vecteur d’initialisation
utilisée pour chiffrer un fichier. \texttt{openssl} utilise un système de mot de passe ce qui est compatible
avec la mémoire de l’utilisateur:
\begin{lstlisting}[language=bash]
$ openssl enc -e -ciphername -in clear -out cipher
\end{lstlisting}

\texttt{openssl} va demander choisir un mot de passe et de le confirmer.


\subsection*{Questions}
\begin{itemize}

\item Chiffrer le fichier \texttt{test1} avec l’AES en mode CTR (128 bits) en utilisant un mot de passe.
Vous obtiendrez un fichier \texttt{res1.enc}
\item Chiffrer une seconde choix le fichier \texttt{test1} avec l’AES en mode CTR (128 bits) en utilisant
le même mot de passe que dans la question précédente. Vous obtiendrez un fichier \texttt{res2.enc}


\item Comparez \texttt{res1.enc} et \texttt{res2.enc}. Le résultat vous surprend-t-il ?
\item Chiffrer une troisième fois \texttt{test1} toujours avec le même mode opératoire de l’AES et le
même mot de passe mais vous allez utiliser l’option \texttt{-p}. Expliquer comment la clef et le vecteur
d’initialisation sont générées.
\item Utiliser \texttt{hexdump} pour comprendre la structure d’un fichier chiffré avec un mot de passe.
Quelle option doit-on donnée à \texttt{openssl} pour obtenir après chiffrement un fichier qui contient
uniquement le texte chiffré ?
\end{itemize}




\section{Hachage}

On peut calculer l’empreinte d’un fichier avec openssl en utilisant les commandes \texttt{dgst}:
\begin{lstlisting}[language=bash]
$ openssl dgst -hashname filename
\end{lstlisting}

avec les options suivantes:
\begin{itemize}
\item  \texttt{-hashname} le nom de la fonction de hachage
\item  \texttt{filename} le nom du fichier
\end{itemize}



\subsection*{Questions}
\begin{itemize}
\item Calculer l’empreinte du fichier \texttt{test1} avec SHA-256 et SHA-512.

\item Calculer les empreintes de  \texttt{l1.ps} et  \texttt{l2.ps} avec MD5. Visualiser ces deux documents. Qu'en pensez vous ?

\item Pour calculer l’empreinte d’une chaîne de caractères, on peut utiliser la commande:
\begin{lstlisting}[language=bash]
$ echo "mystring" | openssl dgst -hashname
\end{lstlisting}

L’empreinte de "toto" est:
\texttt{Ox0b9c2625dc21ef05f6ad4ddf47c5f203837aa32c}
Êtes vous d’accord ? (A quoi faut-il faire attention ?)
\item \texttt{hashcash} est un système de preuve de travail utilise dans Bitcoint. Pour générer de
nouveau bitcoin, il faut réussir à trouver des pré-images pour des empreintes SHA-256 de la forme:
\begin{center}
\texttt{0x000000XXXXXXXXXXXXXXXXXXXXXXXXXXXXXXXXXXXXXXXXXXXXXXXXXXXXXXXXXX}
\end{center}
\end{itemize}

\section{Chiffrement asymétrique}
Nous allons regarder comment utiliser RSA avec \texttt{openssl} La génération des clefs est faites de la
façon suivante:
\begin{lstlisting}[language=bash]
$ openssl genrsa -out filename size
\end{lstlisting}
avec comme paramètres:
\begin{itemize}
\item \texttt{filename} le fichier dans lequel les clefs vont être écrites.
\item \texttt{size} la taille en bits de la clef.
\end{itemize}
Pour inspecter et visualiser les clefs, on dispose de la commande suivante:
\begin{lstlisting}[language=bash]
$ openssl rsa -in filename -text -noout
\end{lstlisting}

Le fichier \texttt{filename} contient la clef publique et de la clef privée: il faut y faire très attention et
ne jamais divulguer la clef privée! Par contre on peut partager avec tout le monde la clef publique.
Pour extraire la clef publique, on dispose de la commande suivante:
\begin{lstlisting}[language=bash]
$ openssl rsa -in filename -pubout -out filename.pub
\end{lstlisting}

Le fichier \texttt{filename.pub} peut être partagé avec n’importe qui !
Pour chiffrer un fichier, on dispose de la commande suivante

\begin{lstlisting}[language=bash]
$ openssl rsautl -encrypt -pubin -in clear -inkey filename.pub -out cipher
\end{lstlisting}

avec comme paramètres:
\begin{itemize}
\item \texttt{clear} le fichier contenant le texte clair,
\item \texttt{cipher} le fichier contenant le texte chiffré.
\end{itemize}

\subsection*{Questions}
\begin{itemize}
\item Tester le chiffrement avec RSA 2048 sur un fichier de votre choix en échangeant par mail avec
un autre groupe vos clefs publiques.
\item Même question pour le fichier contenant 3KB de données aléatoires. Que pensez vous du chiffrement avec RSA ? Comment faire pour être plus efficace ?
\end{itemize}

\end{document}